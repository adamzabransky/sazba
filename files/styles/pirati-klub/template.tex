\usepackage[margin=2.5cm,footskip=1cm,bottom=4cm,top=4cm]{geometry}

% nastavení fontů
\setmainfont[
    Path = files/fonts/LiberationSans/,
    Ligatures=TeX,
    UprightFont = *-Regular,
    BoldFont=*-Bold.ttf,
    ItalicFont=*-Italic.ttf,
    BoldItalicFont=*-BoldItalic]{LiberationSans}
\setmonofont[
    Path = files/fonts/LiberationSans/,
    Ligatures=TeX,
    UprightFont = *-Regular,
    BoldFont=*-Bold.ttf,
    ItalicFont=*-Italic.ttf,
    BoldItalicFont=*-BoldItalic]{LiberationSans}

% Pozadí stránky
\usepackage{wallpaper}
\usepackage{fancyhdr}
\fancypagestyle{mypagestyle}{%
  \fancyhf{}% Clear header/footer
  \renewcommand{\headrulewidth}{0pt}
  \fancyfoot[C]{\small Česká pirátská strana $\bullet$ Řehořova 19, Praha 3 $\bullet$ IČO 71339698 \\
Tel. +420\,775\,978\,550 $\bullet$ Web \url{http://praha.pirati.cz} $\bullet$ E-mail \href{mailto:info@pirati.cz}{info@pirati.cz} $\bullet$ Datová schránka \uline{b2i4r6j} \\
Transparentní účet 2100048174/2010}% Your logo/image
}

\fancypagestyle{empty}{%

  \fancyhf{}% Clear header/footer
  \renewcommand{\headrulewidth}{0pt}
  \fancyfoot[C]{\thepage}% Your logo/image
}
\pagestyle{empty}

% první stránka obsahuje pozadí
\AtBeginDocument
{
   \thispagestyle{mypagestyle}
   \ThisULCornerWallPaper{1}{files/styles/pirati/pozadi.pdf}
}

% Titulek

\renewcommand{\title}[1]{
\vspace*{0.3cm}
\begin{center}
{\LARGE #1}
\end{center}
\hypersetup{
    pdftitle={#1},
    pdfauthor={Česká pirátská strana},
    hidelinks=true,
    pdfborder={0 0 0}
}
}

% Číslování nadpisů

\def\arabicsections{
\titleformat{\section}
{\normalsize\bfseries\center}
{Čl.\,\thesection}{1ex}{}
\renewcommand{\thesection}{\arabic{section}}}

\def\Romansections{
\titleformat{\section}
{\normalsize\bfseries\center}
{\thesection}{1ex}{}
\renewcommand{\thesection}{\Roman{section}.}

\titleformat{\subsection}
{\normalsize\center\emph}
{\thesubsection}{1ex}{}
\renewcommand{\thesubsection}{\Roman{section}.\alph{subsection})}
}


\titlespacing{\section}{0em}{2em}{0.5em}

%%%%%%%%%%% DEFINOVÁNÍ STYLŮ %%%%%%%%%%%%%%%%%
\usepackage{enumerate}
\renewcommand{\labelitemi}{$-$}
